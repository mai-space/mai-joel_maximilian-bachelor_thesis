%----------------------------------------------------------------------------------------
%	Debug options
%----------------------------------------------------------------------------------------
% chktex-file 2
% chktex-file 8
% chktex-file 11
% chktex-file 13
% chktex-file 18
% chktex-file 36
% chktex-file 39
% chktex-file 44
%----------------------------------------------------------------------------------------
\chapter{Conclusion}\label{cha:Conclusion}
% # Why does this section matter?
% # How does it reflect on investigating the gap in Scrum
% # Transition to the next section
% Conclude what have been learned, refer to the research question(s), refer to literature, compare findings
This chapter brings the investigation into the gap between the theory and practice of Scrum to a close. It presents a summary of the findings from the literature review and data collection and provides an overview of the practical implications of the results. The conclusion provides a final synthesis of the research question and the key findings and draws a clear connection between the investigation and its implications for both practice and future research.

% What is the theory-practice gap of Scrum?
% Provide common knowledge about Scrum and the theory-practice gap
% what are its characteristics?
The first research question guiding this thesis is "What is the theory-practice gap of Scrum?". The theory-practice gap of Scrum can be characterized by splitting it up into three dimensions "language", "\gls{methodology}" and "evolution". While the "language" dimension refers to deviations, lack of verbalization and misunderstandings about the \gls{methodology} of Scrum and its effectiveness, the dimension "\gls{methodology}" has attributes like how existing roles and \glspl{method} or processes can be transformed, how the \gls{adoption} of Scrum can be strategically planned and how the \gls{framework} could be scaled up to larger teams and organizations. The third dimension "evolution" visualizes the attributes in which Scrum has been changed up and improved, for example through scaling \glspl{framework} or moving away from prepared \glspl{framework} and embracing the \ac{agile-manifesto} and its \gls{ideology}.  

% what are the perceptions of Scrum practitioners regarding the challenges faced?
% Identify the challenges faced by Scrum practitioners
% Explore the perceptions of Scrum practitioners
% Investigate the impact of the theory-practice gap on organizations and teams
The second research question of this thesis is "What are the perceptions of Scrum practitioners regarding the challenges faced?". This literature review and study have confirmed that the theory-practice gap of Scrum had many implications for organizations and teams. They often correlated with the lack of support, training, involvement and \gls{commitment}. Among the reported issues were the lack of involvement of the employees and the lack of involvement of the \gls{client} in the testing processes. This has led to a reduced alignment of the teams and missed opportunities for greater \gls{client} satisfaction. Another issue reported in the literature and this study is the lack of support through upper management and the lack of \gls{commitment} of other departments than development to the \gls{adoption} of the \gls{framework}, which has led to frustration for the \glspl{developer} and reduced benefits of Scrum.

% what were their solutions tried and what guidelines can be derived from them?
% Provide insights for organizations and practitioners to improve their implementation of Scrum
The third research question providing a goal for this thesis is "what were the solutions tried and what guidelines can be derived from them?". The various solutions tried most often lead to the same conclusions that can be drawn from them. To observe the successful \gls{adoption} of a \gls{framework}, an organization has to listen carefully to the requirements their employees have and commit to the change. Further implications for practitioners are explained in the next section.

\newpage

\section{Guidelines for the adoption of Scrum}\label{sec:Implicationsforpractice}
% # Why does this section matter?
% # How does it reflect on investigating the gap in Scrum
% # Transition to the next section
% Recommend a set of methods or behavior guidelines on how to solve the individual gap
This section provides a summary of the practical implications of the findings for Scrum practitioners and organizations. It provides a clear overview of the recommendations for overcoming the challenges faced in implementing Scrum and offers practical guidance for organizations looking to implement the \gls{methodology} effectively. The implications for practice are based on a thorough analysis of the results and provide actionable recommendations for organizations looking to adopt Scrum.

The evolution of the Scrum \gls{framework} has been the subject of ongoing research, with contributions from the original authors and practitioners in the field. \Gls{adoption} of the Scrum \gls{framework} in organizations has been found to be influenced by a variety of factors, including organizational size, maturity, culture, and knowledge of the \gls{framework}. 

To address these challenges, organizations can prioritize the needs of their employees, consider the input of critics, research available \glspl{methodology} and \glspl{framework}, invest in employee training, and involve employees in the decision-making process. By adopting an inspect-and-adapt cycle, organizations can continuously refine their custom \gls{methodology} and ensure its ongoing value. However, resolving issues of collaboration with \glspl{client} and cultural differences within organizations may require letting go of resistant employees and educating \glspl{client} about the benefits of \ac{asd}. The support of top management is crucial for the success of the \gls{adoption} and \gls{transformation} process, and Scrum Masters and Agile Coaches can provide valuable guidance in this regard.

Scaling Scrum to accommodate larger teams or multiple products can be accomplished through the use of scaling \glspl{framework}. However, it is recommended to first establish a strong foundation in \Gls{agile} \glspl{principle}, potentially through the use of Scrum, before exploring scaling \glspl{framework} and adapting them to fit the specific needs of the organization. Keeping the \gls{methodology} simple and avoiding complexity is crucial in ensuring the effectiveness of the scaling process.

\section{Limitations of the study}\label{sec:Limitationsofthestudy}
% # Why does this section matter?
% # How does it reflect on investigating the gap in Scrum
% # Transition to the next section
% The Interviewee Count is far below what I had anticipated. But the answers given by the Interviewees are interesting enough that I am convinced that the analysis will provide sufficient Inside into this matter, but I am also convinced that a much broader Study of this topic will probably be able to go even more in-depth.
This section highlights the limitations of the investigation into the gap between the theory and practice of Scrum. It provides an overview of the limitations of the \gls{methodology} used in this thesis and the limitations of the results. The conclusion of the limitations of the study provides important context for interpreting the results and understanding the limitations of the investigation.

Despite the ongoing efforts of practitioners and authors to enhance the understanding and address emerging issues, the literature on this subject is constantly evolving and subject to change. As the field of Software Development advances quickly, it is possible that new research and insights may emerge that further shed light on the Scrum theory-practice gap.

The study was designed to gather data through the administration of questionnaires, however, the response rate was lower than anticipated. Despite this, the results obtained were sufficient to identify key areas of concern for organizations and provide support for previous literature on the subject. 

It is recommended that future studies with higher participation and a more diverse sample population be conducted to provide a more comprehensive understanding of the Scrum theory-practice gap.

The results of this study serve as a useful starting point for further research and exploration into the complexities of the Scrum theory-practice gap. While the results provide interesting observations and indicate potential areas for further research, caution must be exercised in interpreting the results and making conclusive statements. A larger sample size or other \glspl{methodology} may be necessary to fully validate the findings of this study.

\section{Future research on the theory-practice gap of Scrum}\label{sec:Implicationsforfutureresearch}
% # Why does this section matter?
% # How does it reflect on investigating the gap in Scrum
% # Transition to the next section
% What were the limitations of the research
% What was done well, what can be done in the future
This section provides areas in which future research can build upon the findings of this investigation and provides direction for future investigations into the gap between the theory and practice of Scrum. The implications for future research are based on a thorough analysis of the results and provide a roadmap for advancing the field of study.

The results of previous studies and the present study raise several questions regarding the \gls{adoption} and implementation of the Scrum \gls{framework}. The following topics warrant further investigation:

\begin{description}[style=nextline]
    \item[The use of external experts in organizations]
    Understanding why organizations rely so heavily on external experts and why their use is so prevalent could provide valuable insights into how organizations can better optimize the hiring process and manage the role of these experts.
    \item[The perceived simplicity of frameworks]
    It is important to understand why organizations treat \glspl{framework} as ready-made solutions and why there is a tendency to rush the process.
    \item[Change management]
    Further research is needed to investigate why change management is not receiving sufficient support from top management and how organizations can address this issue.
    \item[Training in methodology and communication]
    The importance of training in \gls{methodology} and communication needs to be better understood, and the impacts and requirements for training in feedback and communication culture need to be explored.
    \item[Cultural differences and organization size]
    Investigating the impacts of cultural differences inside organizations of different sizes could provide valuable insights into how organizations can better address these differences.
    \item[Sprint length]
    The impact of sprint length on the implementation and success of the Scrum \gls{framework} needs to be studied in greater detail.
    \item[Advantages and disadvantages of fully-featured frameworks]
    A more in-depth analysis of the benefits and disadvantages of fully-featured \glspl{framework} could provide valuable insights into how organizations can optimize their \gls{adoption} and implementation of Scrum.
    \item[Impacts of revisions to the Scrum Guide]
    Understanding the impact of revisions to the Scrum Guide on transitioning organizations is crucial in order to resolve different understandings of Scrum and refine the \gls{framework}.
    \item[Client collaboration]
    Further research is needed to understand why organizations struggle with \gls{client} collaboration and what the reasons for this struggle are. This could provide valuable insights into how organizations can better address these issues and improve collaboration with \glspl{client}.
\end{description}
    
\par\noindent\rule{\textwidth}{0.4pt}

Although Software Development and the \glspl{framework} guiding its processes are always changing, it is worth it to investigate the causes of issues and derived solutions on a regular basis. The body of knowledge around the practices of Software Development and working together collaboratively will evolve in the future and with the increasing usage of artificial intelligence and new opportunities in communication, future researchers may find some of the problems and solutions presented by this thesis to be outdated. As for now, this study has investigated currently faced issues and presented \glspl{guideline} for organizations, on how they can adopt the Scrum \gls{framework} without facing the issues of the theory-practice gap.
