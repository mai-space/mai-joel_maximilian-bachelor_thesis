\chapter{Fazit}
\label{cha:Fazit}
% wird fortlaufend auf der Basis benutzerzentrierter Evaluierung vorangetrieben
% Überblick über den Aufbau der Arbeit, Ergebnisse der einzelnen Kapitel
% Ergebnisse und Forschungsfrage in Beziehung setzen: „Harmonie zwischen den aus dem Thema abgeleiteten Fragestellung(en) und den im Schlussteil ausgewiesenen Ergebnissen, die Antworten zu diesen Fragestellungen geben“
% Ergebnisse in Forschungskontext einordnen, Geltungsbereich kritisch einschätzen (vgl. Winter 2004: 76); selbstkritische Reflexion, Kritikpunkte, Fehlstellen und Beschränkungen
% Schlussfolgerungen, offene Fragen (vgl. Samac, Prenner & Schwetz 2014: 74), Vorschläge für weitere Forschung: „future research“ (vgl. Franck 2004: 199)