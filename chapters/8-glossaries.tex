%----------------------------------------------------------------------------------------
%	Debug options
%----------------------------------------------------------------------------------------
% chktex-file 36
% chktex-file 39
% chktex-file 18
% chktex-file 8
% chktex-file 11
%----------------------------------------------------------------------------------------
%	Usage options
%----------------------------------------------------------------------------------------
% \gls{some}
% \Gls{some}
% \glspl{some}
% \Glspl{some}

\newglossaryentry{client}
{
    name=client,
    plural={clients},
    description={Clients are the contractors of a product. They are also known as business owners/ Product Owners. In some projects, the client is also the customer}
}
\newglossaryentry{customer}
{
    name=customer,
    plural={customers},
    description={Customers are using the product. They are also known as shoppers or users. In some projects, the customer is also the client}
}
\newglossaryentry{developer}
{
    name=developer,
    plural={developers},
    description={"We use the word “developers” in Scrum not to exclude, but to simplify. If you get value from Scrum, consider yourself included. [...] Developers are the people in the Scrum Team that are committed to creating any aspect of a usable Increment each Sprint." -- \citeA[p.~2,6]{Schwaber2020Tsg}}
}
\newglossaryentry{self-managing}
{
    name={self-managing},
    description={People are choosing who, how, and what to work on~\cite[Changes between 2017 and 2020 Scrum Guides]{Schwaber2020SGR}}
}
\newglossaryentry{self-organizing}
{
    name={self-organizing},
    description={People are choosing who and how to do the work~\cite[Changes between 2017 and 2020 Scrum Guides]{Schwaber2020SGR}}
}
\newglossaryentry{agile}
{
    name={agile},
    description={Being able to move quickly and easily. The ability to create and respond to change. In the context of Software Development, it stands for working with commitment to the Manifesto for Agile Software Development which states four core values and twelve principles~\cite{Beck2001MfA}}
}
\newglossaryentry{method}
{
    name={method},
    plural={methods},
    description={In the context of methodology, a method refers to a systematic and structured process used to solve a problem or achieve a goal. A method is a step-by-step approach to a task that helps ensure consistency, accuracy, and repeatability. A methodology is a set of methods and techniques used to approach a specific type of problem or task~\cite{Cooper2018MVP}}
}
\newglossaryentry{principle}
{
    name={principle},
    plural={principles},
    description={In the context of ideology, a principle refers to a fundamental belief or tenet that serves as the basis for a particular political or moral system. Principles in ideology are the cornerstone beliefs that shape the actions, decisions, and values of individuals and groups. They provide a framework for understanding the world and guide individuals in making choices and taking actions~\cite{Cooper2018MVP}}
}
\newglossaryentry{ideology}
{
    name={ideology},
    plural={ideologies},
    description={Doctrine, philosophy, a body of beliefs or principles belonging to an individual or group~\cite{2018IvM}}
}
\newglossaryentry{methodology}
{
    name={methodology},
    plural={methodologies},
    description={A Methodology is the set of conventions that a team agrees to follow. That means that each team will have its methodology, which will be different in either small or large ways from every other team's methodology~\cite{2022WiA}}
}
\newglossaryentry{framework}
{
    name={framework},
    plural={frameworks},
    description={Frameworks were born from a single team's methodology, but they became frameworks when they were generalized to be used by other teams. Frameworks help inform where a team starts with its methodology, but they shouldn't be the team's methodology. The Team will always need to adapt its use of a framework to fit properly in its context~\cite{2022WiA}}
}
\newglossaryentry{mindset}
{
    name={mindset},
    plural={mindsets},
    description={A way of thinking; an attitude or opinion, especially a habitual one~\cite{2018IvM}. \citeA{2022WiA} states, that Agile is a mindset informed by the Agile Manifesto}
}
\newglossaryentry{guideline}
{
    name={guideline},
    plural={guidelines},
    description={Guidelines are a set of recommendations or suggestions that provide direction or advice for a specific situation or task. Guidelines are designed to help individuals make decisions or take actions that are consistent with a particular goal or objective. They are often based on best practices, standards, or established norms and are meant to provide a general framework for decision-making~\cite{CambridgeDictionaryGE}}
}
\newglossaryentry{plan-driven}
{
    name={plan-driven},
    description={A plan-driven approach, also known as a "waterfall" approach, is a methodology used in project management and software development that follows a linear and sequential process. It is based on the principle of careful planning and strict control of the development process. In a plan-driven approach, each stage of the project is completed in full before moving on to the next stage, and changes to the plan are carefully controlled and managed~\cite{InnolutionPDP}}
}
\newglossaryentry{dod}
{
    name={Defintion of Done},
    description={The Definition of Done (DoD) is a formal statement that outlines the necessary criteria for an increment of work to be considered complete and meet the established quality standards for the product. The DoD serves as a shared understanding for all stakeholders, ensuring transparency and consistency in the completion of work.
        At the point when a Product Backlog item satisfies the DoD, it is considered a complete increment. If a Product Backlog item fails to meet the DoD, it cannot be deemed ready for release or presentation at the Sprint Review, and instead must be returned to the Product Backlog for further refinement.
        In cases where the organization has established DoD standards, all Scrum Teams are expected to abide by these minimum requirements. If no such standards exist, the Scrum Team must create a DoD that is appropriate for their specific product.
        All developers must adhere to the established DoD, and in instances where multiple Scrum Teams are collaborating on a product, they must mutually agree on and comply with the same Definition of Done~\cite{Huether2017TDo}}
}
\newglossaryentry{transition}
{
    name={transition},
    plural={transitions},
    description={In the context of framework adoption, transition refers to the process of moving from one framework, method, or approach to another. This can involve changing from one software development framework to another, transitioning from one project management methodology to another, or adopting a new organizational structure or way of working}
}
\newglossaryentry{transformation}
{
    name={transformation},
    plural={transformations},
    description={In the context of framework adoption, transformation refers to a significant change or overhaul of a framework, method, or approach. It is often a more comprehensive change than a transition and involves a complete overhaul of the existing framework to create a new, more effective one. Transformation can include changes to processes, structure, culture, and technology}
}
\newglossaryentry{adoption}
{
    name={adoption},
    plural={adoptions},
    description={In the context of framework adoption, adoption refers to the process of incorporating a new framework, method, or approach into an organization or system. This process involves a series of steps, including planning, implementation, and stabilization, and is designed to ensure that the new framework is effectively integrated into the existing system and can be used to achieve desired outcomes}
}
\newglossaryentry{adaptation}
{
    name={adaptation},
    plural={adaptations},
    description={In the context of framework adoption, adaptation refers to the process of modifying a framework, method, or approach to better suit the specific needs of an organization or system. This process involves adjusting the framework to align with the existing processes, structure, culture, and technology of the organization so that it can be effectively integrated and used to achieve desired outcomes}
}
\newglossaryentry{scope-creep}
{
    name={scope-creep},
    description={Sometimes known as "requirement creep" or even "feature creep". It describes the tendency to increase the requirements of a product over a project's lifecycle~\cite{ProjectManagementQualification2019SCD}}
}
\newglossaryentry{commitment}
{
    name={commitment},
    description={In the context of Scrum, a commitment refers to a shared understanding between team members regarding what they plan to accomplish during a specific time frame, usually a sprint. In Scrum, the team members, including the product owner, development team, and Scrum Master, work together to make a collective commitment to deliver a specific set of features or functionalities by the end of the sprint~\cite{Schneider2018WbC}}
}
\newglossaryentry{ownership}
{
    name={ownership},
    description={In the context of software development, ownership refers to the responsibility and accountability that a person or group has for a specific aspect of a software project. This can include ownership of specific code modules, features, or components, as well as ownership of processes, testing, documentation, or other aspects of the project~\cite{scrumdictionary.comPO}}
}
